\documentclass{beamer}

\mode<presentation> {
\usetheme{Madrid}
}

\usepackage{graphicx} 
\usepackage{booktabs}
\usepackage[utf8]{inputenc}

\title[Le pouvoir des Réseaux Sociaux]{TRANSFORMATION DE L’AGRICULTURE}
\subtitle{Le pouvoir des Réseaux Sociaux}

\author{François KOBON}
\institute[FIRCA]
{
Fonds Interprofessionnel pour la Recherche et le Conseil Agricoles \\
\medskip
\textit{fkobon@firca.ci}
}
\date{\today}

\begin{document}

\begin{frame}
\titlepage 
\end{frame}

\AtBeginSection[]
{
\begin{frame}
 \frametitle{Sommaire}
 \tableofcontents[hideothersubsections]
\end{frame}
}

\section{Les Médias Sociaux : Aperçu et initiation}
 \subsection{Définition des médias sociaux}
  \begin{frame}
   \frametitle{Définition}
   Les médias sociaux désignent un ensemble de services/applications et technologies impliquant :
   \textcolor{green}{Des conversations et des interactions sociales (entre les personnes) à travers une plateforme virtuelle sur le web/internet en toute mobilité}
  \end{frame}
  
  \subsection{Ce qu’il faut retenir...}
   \begin{frame}
    \frametitle{Ce qu'il faut retenir...}
    \textbf{Conversations sociales :} Une interaction entre les personnes dans les 2 sens
    \newline \newline   
    \textbf{Plateforme virtuelle :} Une plateforme non physique en ligne sur internet ou intranet (hors connexion)
    \newline \newline
    \textbf{Mobilité :} qui concerne à la fois les terminaux mais également les situations de mobilité
   \end{frame}

  \subsection{Outils au service des médias sociaux}
   \begin{frame}
    \frametitle{Les outils au service des médias sociaux}
    \begin{center}
     %\includegraphics[scale=0.4]{res/iot.png}
    \end{center}
   \end{frame}

  \subsection{Les familles de médias Sociaux}
   
   \subsubsection{Les médias sociaux de publication}
   \begin{frame}
    \frametitle{Les médias sociaux de publication}
    Comprennent notamment les blogs et sites participatifs comme les wikis. Ils touchent généralement une communauté de lecteurs étendue.
   \end{frame}

   \subsubsection{Les médias sociaux de partage}
   \begin{frame}
    \frametitle{Les médias sociaux de partage}
    Rassemblent de nombreux réseaux sociaux de contenu. Ils ont pour vocation de proposer un contenu destiné à être partagé par la communauté.
   \end{frame}

   \subsubsection{Les médias sociaux de réseautage}
   \begin{frame}
    \frametitle{Les médias sociaux de réseautage}
    Sont surtout représentés par les réseaux sociaux de contact. Les internautes qui s’y connectent souhaitent interagir directement avec d’autres utilisateurs, et élargir leurs cercles de connaissances selon des objectifs personnels ou professionnels.
   \end{frame}

   \subsubsection{Les médias sociaux de discussion}
   \begin{frame}
    \frametitle{Les médias sociaux dediscussion}
    Comptent notamment dans leurs rangs les forums de discussions et les messageries instantanées. Les conversations s’organisent généralement autour d’affinités personnelles ou à des fins professionnelles.
   \end{frame}
   
  \subsection{Recapitulons}
   \begin{frame}
    \frametitle{Recapitulons}
    Les médias sociaux offrent une nouvelle vision d’Internet qui revient à \textbf{considérer l’internaute comme \textcolor{brown}{ACTEUR} et partie prenante de l’information}, dans un\textbf{\textcolor{brown}{DIALOGUE}}plus que dans un monologue.
   \end{frame}

  \subsection{Les médias traditionnels VS médias sociaux}
   \begin{frame}
   \frametitle{Les médias traditionnels VS médias sociaux}
   \end{frame}

  \subsection{Comparaison}
   \begin{frame}
    \frametitle{Comparaison}
    \begin{center}
     %\includegraphics[scale=0.4]{res/comparaison.png}
    \end{center}
   \end{frame}

  \subsection{Avantages des médias sociaux }
   \begin{frame}
    
   \end{frame}


\section{Les Réseaux Sociaux : Principes et Défis}

\subsection{Principes des Réseaux Sociaux}
\begin{frame}
\frametitle{Principes des Réseaux Sociaux}
 Il s’agit d’une communauté d’individus ou d’organisations mise en relation virtuelle etrassembléeen fonction de centres d’intérêts communs
\end{frame}
\begin{frame}
\frametitle{Principes des Réseaux Sociaux}
\begin{itemize}
\item Adhèrent à une communauté 
\item Créent ensemble du contenu Web, 
\item Partagent en ligne les informations
\item Collectent des informations
\item Font des commentaires
\item Donnent leur avis
\end{itemize}
\end{frame}
\begin{frame}
\frametitle{Principes des Réseaux Sociaux}
\begin{itemize}
\item Donner un visibilité sur vos actualités, vos produits et votre secteur.
\item Créer votre propre audience.
\item Provoquer l’échange et dynamiser votre communication.
\item Adopter un discours plus humain et interactif.
\item Créer un cadre d’échange entre les acteurs clés
\end{itemize}
\end{frame}

\subsection{Aperçu des Réseaux Sociaux}
\begin{frame}
\frametitle{Aperçu des Réseaux Sociaux}
Image 1
\end{frame}
\begin{frame}
\frametitle{Aperçu des Réseaux Sociaux}
Image 2
\end{frame}
\begin{frame}
Images 2
\end{frame}

\section{La stratégie sociale et digitale}
\subsection{La stratégie digitale}
\begin{frame}
Aujourd’hui, ne pas utiliser les réseaux sociaux dans sa stratégie de communication globale semble improbable. La proximité, la fidélisation et l’engagement qu’engendrent les media sociaux – ou social media – peuvent constituer un véritable boost en terme d’acquisition de notoriété, de crédibilité, de promotion 
\end{frame}

\section{Les Réseaux Sociaux appliqués à l'Agriculture}
\subsection{Défis pour le secteur agricole}
\begin{frame}
\frametitle{Défis pour le secteur agricole}
L’agriculture est une activité économique où l’information et la communication sont essentielles. 
\end{frame}

\section{Valorisation des résultats FIRCA sur les réseaux sociaux}
\subsection{Défis pour le FIRCA}
\begin{frame}
\frametitle{Défis pour le FIRCA}
Sur le point d'amorcer sa deuxième phase, la stratégie de communication du FIRCA se concentre sur \textcolor{red}{\textbf{l’intensification de la diffusion et l'adoption de technologies agricoles améliorées}} dans les secteurs agricoles prioritaires des pays bénéficiaires du programme.
\end{frame}
\subsection{Défis majeurs}
\begin{frame}
\frametitle{Défis majeurs}
\begin{itemize}
\item Faible visibilité du FIRCA malgré les résultats significatifs de la 1ère phase
\item Absence d’approche participative et inclusive dans la conception de la stratégie de communication (producteurs, bénéficiaires)
\item Faible niveau de communication avec les partenaires financiers et techniques
\item Absence de Cadre d’échanges des acteurs de la chaîne de valeur et autres partie prenantes
\item Absence de branding du FIRCA comme HUB DE CONNAISSANCE
\end{itemize}
\end{frame}
\subsection{Cas de Facebook}
\begin{frame}
\frametitle{Utilité de Facebook pour le FIRCA}
\begin{itemize}
\item Publier des photos, des vidéos de démonstrations, des fiches techniques commentées des technologies 
\item Partager l'information sur les bonnes pratiques agricoles.
\item Se connecter avec des pages traitant des sujets relatifs à l'agriculture.
\item Poster des success stories et des vidéos des bénéficiaires 
\item Créer des groupes d'échanges d'acteurs de la chaîne de valeur agricole
\item Créer des événements 
\end{itemize}
\end{frame}
\begin{frame}
\frametitle{En gros, Facebook pour le FIRCA c'est de :}
Fournir du contenu qui permettra de combler certaines lacunes au chapitre de l’information, de rectifier les perceptions erronées et de raconter des histoires de réussite des filières et du secteur.
\newline \newline
Accuser réception des questions et commentaires et y répondre.
\newline \newline
Suivre et soutenir (en utilisant les mentions "J’aime") les organismes et les personnes qui jouent des rôles clés dans l’industrie agricole et fournir un contenu digne d’intérêt dans le cadre des médias sociaux.
\end{frame}
\subsection{Comment choisir sa page Facebook}
\begin{frame}{Comment choisir sa page Facebook}
    Images
\end{frame}


\section{Travaux Pratiques}

\end{document}